%% LyX 2.3.7 created this file.  For more info, see http://www.lyx.org/.
%% Do not edit unless you really know what you are doing.
\documentclass[french]{article}
\usepackage[T1]{fontenc}
\usepackage[utf8]{inputenc}
\usepackage{babel}
\makeatletter
\addto\extrasfrench{%
   \providecommand{\og}{\leavevmode\flqq~}%
   \providecommand{\fg}{\ifdim\lastskip>\z@\unskip\fi~\frqq}%
}

%\newcommand{\oper}[1]{\sqrt{#1}}
\makeatother
\newcommand{\commandevide}{}
\begin{document}

\title{Exemple de fichier \LaTeX{} long et complexe}
\author{Philippe Helluy}
\maketitle
\begin{abstract}
Ceci est un exemple étendu de fichier \LaTeX{} pour tester la capacité de découpage automatique de trsltx.
Il contient du texte, des mathématiques, des références et doit être suffisamment long.
\end{abstract}

\section{Introduction}
L'objectif du projet trsltx est de réaliser un outil pour traduire
des documents \LaTeX{} d'une langue à une autre en conservant la syntaxe.
Nous allons ici tester si le logiciel est capable de gérer de longs passages sans marqueurs explicites.

Pour cela, nous introduisons plusieurs concepts mathématiques.
\begin{itemize}
\item Le document peut contenir des formules en ligne comme $x=\sqrt{3}$ ou $e^{i\pi} + 1 = 0$.
\item Il peut aussi contenir des formules hors ligne.
\end{itemize}

\section{Développements Mathématiques}

Considérons l'équation de la chaleur en dimension 1 :
\begin{equation}
\frac{\partial u}{\partial t} - \alpha \frac{\partial^2 u}{\partial x^2} = 0 \label{eq_chaleur}
\end{equation}
où $u(x,t)$ est la température à la position $x$ et au temps $t$.

Nous pouvons discrétiser cette équation avec un schéma aux différences finies.
Soit $u_i^n \approx u(x_i, t_n)$. Une approximation explicite est donnée par :
\begin{equation}
\frac{u_i^{n+1} - u_i^n}{\Delta t} - \alpha \frac{u_{i+1}^n - 2u_i^n + u_{i-1}^n}{\Delta x^2} = 0
\end{equation}

Cela nous amène à étudier la stabilité du schéma. Comme vu dans l'équation (\ref{eq_chaleur}), le coefficient de diffusion $\alpha$ joue un rôle crucial.

Ajoutons un peu de texte pour augmenter la longueur du fichier. La simulation numérique des équations aux dérivées partielles est un vaste sujet.
Il existe de nombreuses méthodes : volumes finis, éléments finis, méthodes spectrales.
Chacune a ses avantages et ses inconvénients en termes de précision et de coût de calcul.

\section{Algèbre Linéaire}

Les systèmes discrétisés conduisent souvent à des problèmes matriciels de la forme $Ax=b$.
\begin{equation}
\mathbf{A}=\left[\begin{array}{cc}
1 & 2\\
2 & 5
\end{array}\right].\label{eq_matrice} 
\end{equation}
La matrice $\mathbf{A}$ dans (\ref{eq_matrice}) est symétrique définie positive.
Son inverse peut être calculée analytiquement ou numériquement.

\subsection{Détails supplémentaires}
Pour tester la robustesse du découpage, ajoutons encore du contenu.
Supposons que nous voulions calculer les valeurs propres de $\mathbf{A}$.
Les valeurs propres $\lambda$ sont solutions de $\det(A - \lambda I) = 0$.
Ici, le polynôme caractéristique est :
\[
(1-\lambda)(5-\lambda) - 4 = 0
\]
Soit $\lambda^2 - 6\lambda + 1 = 0$.
Les solutions sont $\lambda = \frac{6 \pm \sqrt{36-4}}{2} = 3 \pm \sqrt{8}$.

\section{Conclusion}
Nous avons maintenant un texte suffisamment long, avec plusieurs sections, des équations mathématiques complexes et des références croisées.
Le logiciel trsltx devrait être capable d'insérer automatiquement des points de coupure entre les sections ou les paragraphes si la longueur dépasse la limite fixée (par défaut 1000 caractères).

Espérons que la traduction en anglais sera de bonne qualité et que la structure \LaTeX{} sera préservée !

%trsltx-begin-ignore
\begin{thebibliography}{1}
\bibitem{tutu}Un auteur, \emph{Un titre}, une revue, 2021.
\end{thebibliography}
%trsltx-end-ignore

\commandevide
\end{document}
